\documentclass{article} % articleというドキュメントクラス(文書の形式)を使う
\begin{document} % 文書を始める

\begin{large}
\noindent
\textbf{Name:} Erdenebat Battseren \: \textbf{Student ID:} 20B60029
\newline
\textbf{The usage of semiconductors and its impact in the world}
\end{large}  
\vspace{0.2 cm}
\noindent
\newline
The usage of semiconductor chips is one of the most significant technological breakthroughs. In 1947, A silicon chip that contains integrated circuits is in our smartphones, computers, and cars. The impact of the semiconductor chip is so significant that the shortage caused by the COVID-19 pandemic is affecting the whole world. In this short essay, let us delve into the reasons why the usage of semiconductors is crucial and the impacts are so huge in our daily lives. 
\vspace{0.2 cm}
\newline
Without semiconductor chips, our daily life gadgets will not work. First of all, let us distinguish the different terminologies. A semiconductor material itself is not an invention. A common example of a semiconductor material is silicon, which sometimes conducts electricity and sometimes does not based on conditions. Microchips use integrated circuits that are made of semiconductors. Processors like CPUs are made of a series of microchips that can do complex computations. These microchips and processes like CPUs are integrated into our daily gadgets and
\vspace{0.2 cm}
\newline
With technological advancements, the world's demand for better-integrated circuits has been increasing so is the consumption of semiconductors. The demand is so high that there is an actual law that shows the growth of semiconductor usage over time. Moore's law says that roughly every two years, the number of transistors on microchips will double. Basically saying that computational progress will become significantly faster, smaller, and more efficient over time.
\vspace{0.2 cm}
\newline
Semiconductor chip density is becoming more and more packed each year. with more technological advancements, 
 
\begin{thebibliography}{9}
    \bibitem{texbook}
    Donald E. Knuth (1986) \emph{The \TeX{} Book}, Addison-Wesley Professional.
    
    \bibitem{lamport94}
    Leslie Lamport (1994) \emph{\LaTeX: a document preparation system}, Addison
    Wesley, Massachusetts, 2nd ed.
    \end{thebibliography}



\end{document} % 文書を終わる